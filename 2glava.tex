\section{Применение методических подходов к оценке стоимости недвижимости}
\subsection{Доходный подход к оценке недвижимости}
\subsubsection{Прямая капитализация дохода}

Свободный земельный участок под магазином оценен в 542000 руб. 
Ставка дохода для аналогичных объектов торговли определена в 28\%.
Размер ежегодного чистого операционного дохода при ведении торгового бизнеса в данном здании составляет 848000 руб.
Продолжительность экономической жизни здания 50 лет.
Определить стоимость здания магазина.

Решение: Общая формула \ref{26}.
\begin{equation}\label{26}
V_B = \dfrac{NOI_O - V_B \times R_B}{R_L}
\end{equation}

1. Определяется доход, относимый к земле:
$ NOI_L = V_L \times R_L = 542 \times 0,28 = 151,76 $

2. Находится доход, относимый к зданию:
$ NOI_B = NOI_O - NOI_L = 848 - 151,76 =696,24 $

3. Рассчитывается ставка капитализации для здания магазина (по методу Ринга форм. 27):

4. Капитализируется доход, приносимый зданием, в его стоимость:
590000
V B 
 4214286 руб.