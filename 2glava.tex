\section{Применение методических подходов к оценке стоимости недвижимости}
\subsection{Доходный подход к оценке недвижимости}
\subsubsection{Прямая капитализация дохода}

Свободный земельный участок под магазином оценен в 542000 руб. 
Ставка дохода для аналогичных объектов торговли определена в 28\%.
Размер ежегодного чистого операционного дохода при ведении торгового бизнеса в данном здании составляет 848000 руб.
Продолжительность экономической жизни здания 50 лет.
Определить стоимость здания магазина.

Решение: Общая формула \ref{26}.
\begin{equation}\label{26}
V_B = \dfrac{NOI_O - V_B \times R_B}{R_L}
\end{equation}

1. Определяется доход, относимый к земле:

$ NOI_L = V_L \times R_L = 542 \times 0,28 = 151,76  \ \text{тыс. руб.}$

2. Находится доход, относимый к зданию:

$ NOI_B = NOI_O - NOI_L = 848 - 151,76 =696,24 \ \text{тыс. руб.} $

3. Рассчитывается ставка капитализации для здания магазина (по методу Ринга форм. \ref{27}):

\begin{equation}\label{27}
R_{\text{Ринга}} = on + of = on + \dfrac{100\%}{n}
\end{equation}

$ R_B = 28\% + \dfrac{100\%}{50}  = 28\% +2\% = 30\% $

4. Капитализируется доход, приносимый зданием, в его стоимость:

$ V_B = \dfrac{696,24}{0,30} = 2320,8 \ \text{тыс. руб.} $

Ответ: Стоимость здания магазина составит 2320,8 тыс. рублей

\subsubsection{Дисконтирование будущих доходов (денежных потоков)}

В соответствии с инвестиционным проектом, вкладывая 2 млн. руб. в покупку оборудования сейчас, владелец кафе в течение последующих 5 лет получает годовой доход, приведенный в таблице \ref{problem8}.

Установить:
\begin{enumerate}
	\item  Какова величина текущей стоимости доходов?
	\item  Будет ли проект окупаемым в течение 5 лет, если ставка дохода составит 20\% годовых?
\end{enumerate}

\begin{table}
	\small
	\centering
	\caption{Годовой доход объекта недвижимости, тыс. руб.}
	\label{problem8}
	\setlength{\extrarowheight}{1.2mm}
	\begin{tabularx}{\textwidth}{|p{2.37cm}|K{2.3cm}|K{2.3cm}|K{2.3cm}|K{2.3cm}|K{2.3cm}|}
		\hline
		Годы&1&2&3&4&5 \\ \hline
		Доход&290&375&620&840&1030 \\ \hline
	\end{tabularx}
\end{table}


Решение:

1. Текущая стоимость ежегодного потока дохода на сегодняшний день --- момент вкладывания денег (форм. 31) (без продажи объекта в будущем) составит:

\begin{equation}\label{31}
V=\sum\limits_{n=1}^n \dfrac{NOI_n}{(1+i)^n} + \dfrac{FV_B}{(1+i)^m}
\end{equation}

$ V = \dfrac{290}{1+0,2}+ \dfrac{375}{(1+0,2)^2} + \dfrac{620}{(1+0,2)^3} +\dfrac{840}{(1+0,2)^4}+ \dfrac{1030}{(1+0,2)^5} = 1679,91\ \text{тыс. руб.} $

2. Чистая текущая стоимость дохода от инвестиций равна $-320,09 $тыс. руб. ($1679,91 - 2000$), следовательно, проект покупки оборудования на рассматриваемых условиях не окупится.

\subsection{Сравнительный подход к оценке недвижимости}
\subsubsection{Метод сравнения продаж}

Требуется определить рыночную стоимость жилого дома общей площадью $350 \text{м}^2$.
Жилой дом (оцениваемый объект) имеет газон.
Имеется информация по сделкам купли-продажи четырех аналогичных объектов в рассматриваемом загородном районе.
Площадь земельных участков объекта оценки и аналогов одинаковая --- 12 соток.
Характеристики оцениваемого объекта и объектов сравнения (аналогов) приведены в таблице \ref{problem9}.

\begin{table}
	\small
	\centering
	\caption{Информация для проведения оценки}
	\label{problem9}
	\setlength{\extrarowheight}{1.2mm}
	\begin{tabularx}{\textwidth}{|p{3.3cm}|K{2.57cm}|K{2cm}|K{2cm}|K{2cm}|K{2cm}|}
		\hline
		\multirow{2}{\linewidth}{\centering Характеристика объекта} &  \multirow{2}{\linewidth}{\centering Оцениваемый объект} &  \multicolumn{4}{c|}{Объекты сравнения (аналоги) } \\ \cline{3-6}
																&	        &  I       &   II        & III      &IV \\ \hline
		Цена продажи, тыс. руб.         &$ ? $        &  $ 3950 $ &   3350   & 2800  & 3200\\ \hline
		Площадь, $\text{м}^2$              &  350  &    400 & 400     &     350  & 350\\ \hline
		Газон                                             &  $ + $         &    $ + $     &     $  + $     &  $ -  $    &$ + $ \\ \hline
		Водопровод 								   &    $  - $      &    $  + $      &       $ - $     &   $ + $    & $ + $\\ \hline
	\end{tabularx}
\end{table}

\begin{center}
	Решение
\end{center}

В качестве единицы сравнения принимается цена продажи единого объекта (дома с земельным участком).
Расчеты представлены в таблице \ref{problem9-2}.

Объяснения вносимых поправок.

Корректировка цен продаж объектов-аналогов производится по методу анализа парных продаж.
При внесении поправок применяются указанные выше правила.

\begin{table}[!hb]
	\small
	\centering
	\caption{Корректировка цен продаж объектов-аналогов, тыс. руб.}
	\label{problem9-2}
	\setlength{\extrarowheight}{1.2mm}
	\begin{tabularx}{\textwidth}{|p{3.3cm}|K{2.57cm}|K{2cm}|K{2cm}|K{2cm}|K{2cm}|}
		\hline
		\multirow{2}{\linewidth}{\centering Характеристика объекта} &  \multirow{2}{\linewidth}{\centering Оцениваемый объект} &  \multicolumn{4}{c|}{Объекты сравнения (аналоги) } \\ \cline{3-6}
		&	               &  I            &   II        & III     &IV \\ \hline
		Цена продажи, тыс. руб.         &$ ? $        &  $ 3950 $ &   3350   & 2800  & 3200\\ \hline
		Площадь, $\text{м}^2$              &  350      &    400 & 400     &     350  & 350\\ \hline
		Поправка за различие в %
		площади, тыс. руб.            &       &      $-750$     &     $-750$      &     0         &          0   \\ \hline
		Газон                                             &  $ + $ &    $ + $&  $  + $&  $ -  $    &$ + $ \\ \hline
		Поправка за наличие %
		газона, тыс. руб.                         &          &         0    &     0       &         $+400$      & 0 \\ \hline
		Водопровод 								   &    $  - $      &    $  + $      &       $ - $     &   $ + $    & $ + $\\ \hline
		Поправка на водопровод, %
		тыс. руб. 										&          &     $  -600 $        &    0  &    $ -600  $   & $ -600 $ \\ \hline
		Скорректирован-ная цена, тыс. руб.   &       &     2600   &   2600    &   2600   &     2600     \\ \hline
		Количество внесенных поправок, ед.     &     &   2    & 1   &  2   &  1     \\ \hline
		Абсолютное значение	поправок, тыс. руб.   &    &    $ -1350 $   &  $ -750  $    &   $ -200 $    &   $ -600  $          \\ \hline
		Рыночная стоимость объекта, тыс. руб.       &   2600    &       &       &       &       \\ \hline
	\end{tabularx}
\end{table}

1) Площадь жилого дома. Размеры жилых домов представлены величинами 350 и 400 кв. м.
Величина поправки за различие в 50 кв.м. площади жилого дома определяется сравнением I и IV объектов (эти объекты различаются по площади,
но схожие по другим характеристикам): $ 3950 - 3200 = 750 $ тыс. руб.
Поправка вносится с соответствующим знаком в зависимости от разницы в площадях оцениваемого жилого дома и объектов-аналогов.
Для II и II объектов сравнения, имеющих площадь 400 кв. м, поправка имеет знак <<$ - $>>.
При равенстве характеристик с объектом оценки поправка равна нулю.

2) Газон.
Величина поправки за наличие газона определяется сравнением объектов III и IV: $ 3200 - 2800 = 400 $ тыс. руб.
У III-го объекта-аналога газон отсутствует (как и у объекта оценки), поэтому поправка для него вносится со знаком  <<$+$>>.

3) Водопровод.
Корректировка цен продаж на разницу в наличии водопровода проводится путем сравнения объектов I и II: $ 3950 - 3350 = 600  $тыс. руб.
Поправка в цену продаж I, III и IV объекта отрицательная, т.к. у оцениваемого объекта водопровод отсутствует.

Таким образом, в качестве рыночной стоимости объекта оценки принимается скорректированная цена продажи, полученная одинаковой по всем четырем объектам-аналогам, в размере 2600 тыс. руб. (мода --- величина, встречающаяся наибольшее количество раз).



Ответ: рыночная стоимость оцениваемого жилого дома составляет 2600 тыс. руб.




