\section{Применение методических подходов к оценке стоимости недвижимости}
\subsection{Доходный подход к оценке недвижимости}
\subsubsection{Прямая капитализация дохода}

Свободный земельный участок под магазином оценен в 542000 руб. 
Ставка дохода для аналогичных объектов торговли определена в 28\%.
Размер ежегодного чистого операционного дохода при ведении торгового бизнеса в данном здании составляет 848000 руб.
Продолжительность экономической жизни здания 50 лет.
Определить стоимость здания магазина.

Решение: Общая формула \ref{26}.
\begin{equation}\label{26}
V_B = \dfrac{NOI_O - V_B \times R_B}{R_L}
\end{equation}

1. Определяется доход, относимый к земле:

$ NOI_L = V_L \times R_L = 542 \times 0,28 = 151,76  \ \text{тыс. руб.}$

2. Находится доход, относимый к зданию:

$ NOI_B = NOI_O - NOI_L = 848 - 151,76 =696,24 \ \text{тыс. руб.} $

3. Рассчитывается ставка капитализации для здания магазина (по методу Ринга форм. \ref{27}):

\begin{equation}\label{27}
R_{\text{Ринга}} = on + of = on + \dfrac{100\%}{n}
\end{equation}

$ R_B = 28\% + \dfrac{100\%}{50}  = 28\% +2\% = 30\% $

4. Капитализируется доход, приносимый зданием, в его стоимость:

$ V_B = \dfrac{696,24}{0,30} = 2320,8 \ \text{тыс. руб.} $

Ответ: Стоимость здания магазина составит 2320,8 тыс. рублей

\subsubsection{Дисконтирование будущих доходов (денежных потоков)}

В соответствии с инвестиционным проектом, вкладывая 1,5 млн. руб. в покупку оборудования сейчас, владелец кафе в течение последующих 5 лет получает годовой доход, приведенный в таблице 3.
Установить:
1. Какова величина текущей стоимости доходов?
2. Будет ли проект окупаемым в течение 5 лет, если ставка дохода составит 18\% годовых?

\begin{table}
	\small
	\caption{Годовой доход объекта недвижимости, тыс. руб.}
	\begin{tabularx}{\textwidth}{|p{2cm}|K{2cm}|K{2cm}|K{2cm}|K{2cm}|K{2cm}|}
		Годы&1&2&3&4&5 \\ \hline
		Доход&290&375&620&840&1030 \\ \hline
	\end{tabularx}
\end{table}








