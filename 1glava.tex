\section{Стоимость денег во времени, функции сложного процента}
\subsection{Накопленная сумма денежной единицы}

Определить, какая сумма будет накоплена на счете к середине 17-го года (n=16,5), если сегодня внести на счет, приносящий 8\% годовых, 3100 руб. Начисление процентов осуществляется ежедневно.

Решение:

Начисление процентов осуществляется более чем раз в год (форм. \ref{1}).
\begin{equation}\label{1}
FV = PV \times(1+\frac{i}{k})^{n \times k},
\end{equation}

$ FV = 3100 \times (1+\frac{0,08}{365})^{16.5 \times 365} =11602,93\  \text{руб.} $

Ответ: через 16,5 лет на счете будет накоплено 11 602 рубля 93 копейки.

\subsection{Текущая стоимость единицы (реверсии)}

Определить текущую стоимость 4450 руб., которые будут получены через 12 лет при 15\% ставке дисконта. Начисление процентов осуществляется в конце каждого квартала.

Решение:

Начисление процентов осуществляется в конце каждого квартала (форм. \ref{2}).
\begin{equation}\label{2}
PV = FV \times \dfrac{1}{(1+\frac{i}{k})^{n \times k}},
\end{equation}

$ PV = 4450 \times \dfrac{1}{(1+\frac{0,15}{4})^{12 \times 4}} = 760,21\  \text{руб.}   $
	
Ответ: Текущая стоимость составит 760 рублей 21 копейку.

\subsection{Накопление денежной единицы за период}

Определить сумму, которая будет накоплена на счете, приносящем 7\% годовых к концу 22-го месяца, если ежемесячно откладывать на счет 3600 руб. Платежи осуществляются: а) в начале каждого месяца; б) в конце каждого месяца.

Решение:

a) Платежи осуществляются в начале каждого месяца (форм. \ref{3}).

\begin{equation}\label{3}
FV = PMT \times \left[\dfrac{\left(1+\frac{i}{k}\right)^{n \times k +1}-1}{\frac{i}{k}}-1 \right],
\end{equation}

$ FV = 3600\times \left[\dfrac{\left(1+\frac{0,07}{12}\right)^{22+1}-1}{\frac{0,07}{12}}-1 \right] = 84736,42 \  \text{руб.}   $

б) Платежи осуществляются в конце каждого месяца (форм. \ref{4}).
\begin{equation}\label{4}
FV =PMT \times \dfrac{(1+\frac{i}{k})^{n \times k}-1}{\frac{i}{k}},
\end{equation}

$ FV =3600 \times \dfrac{(1+\frac{007}{12})^{22}-1}{\frac{0,07}{12}} = 84244,99 \  \text{руб.} $


