\section{Стоимость денег во времени, функции сложного процента}
\subsection{Накопленная сумма денежной единицы}

Определить, какая сумма будет накоплена на счете к середине 17-го года (n=16,5), если сегодня внести на счет, приносящий 8\% годовых, 3100 руб. Начисление процентов осуществляется ежедневно.

Решение:

Начисление процентов осуществляется более чем раз в год (форм. \ref{1}).
\begin{equation}\label{1}
FV = PV \times(1+\frac{i}{k})^{n \times k},
\end{equation}

$ FV = 3100 \times (1+\frac{0,08}{365})^{16.5 \times 365} =11602,93\  \text{руб.} $

Ответ: через 16,5 лет на счете будет накоплено 11 602 рубля 93 копейки.

\subsection{Текущая стоимость единицы (реверсии)}

Определить текущую стоимость 4450 руб., которые будут получены через 12 лет при 15\% ставке дисконта. Начисление процентов осуществляется в конце каждого квартала.

Решение:

Начисление процентов осуществляется в конце каждого квартала (форм. \ref{2}).
\begin{equation}\label{2}
PV = FV \times \dfrac{1}{(1+\frac{i}{k})^{n \times k},
\end{equation}

$$ PV = 4450 \times \dfrac{1}{(1+\frac{0,15}{4})^{12 \times 4} = 750,21\  \text{руб.}   $$
	
Ответ: Текущая стоимость составит 750 рублей 21 копейку.

\subsection{Накопление денежной единицы за период}






