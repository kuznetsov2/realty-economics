\section{Стоимость денег во времени, функции сложного процента}
\subsection{Накопленная сумма денежной единицы}

Определить, какая сумма будет накоплена на счете к середине 17-го года (n=16,5), если сегодня внести на счет, приносящий 8\% годовых, 3100 руб. Начисление процентов осуществляется ежедневно.

Решение:
Начисление процентов осуществляется более чем раз в год (форм. \ref{1}).
\begin{equation}\label{1}
FV = PV \times(1+\frac{i}{k})^{n \times k},
\end{equation}

$ FV = 3100 \times (1+\frac{0,08}{365})^{16.5 \times 365} =11602,93 \text{руб.} $

