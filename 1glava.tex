\section{Стоимость денег во времени, функции сложного процента}
\subsection{Накопленная сумма денежной единицы}

Определить, какая сумма будет накоплена на счете к середине 17-го года (n=16,5), если сегодня внести на счет, приносящий 8\% годовых, 3100 руб. Начисление процентов осуществляется ежедневно.

Решение:

Начисление процентов осуществляется более чем раз в год (форм. \ref{1}).
\begin{equation}\label{1}
FV = PV \times(1+\frac{i}{k})^{n \times k},
\end{equation}

$ FV = 3100 \times (1+\frac{0,08}{365})^{16.5 \times 365} =11602,93\  \text{руб.} $

Ответ: через 16,5 лет на счете будет накоплено 11 602 рубля 93 копейки.

\subsection{Текущая стоимость единицы (реверсии)}

Определить текущую стоимость 4450 руб., которые будут получены через 12 лет при 15\% ставке дисконта. Начисление процентов осуществляется в конце каждого квартала.

Решение:

Начисление процентов осуществляется в конце каждого квартала (форм.~ \ref{2}).
\begin{equation}\label{2}
PV = FV \times \dfrac{1}{(1+\frac{i}{k})^{n \times k}},
\end{equation}

$ PV = 4450 \times \dfrac{1}{(1+\frac{0,15}{4})^{12 \times 4}} = 760,21\  \text{руб.}   $
	
Ответ: Текущая стоимость составит 760 рублей 21 копейку.

\subsection{Накопление денежной единицы за период}

Определить сумму, которая будет накоплена на счете, приносящем 7\% годовых к концу 22-го месяца, если ежемесячно откладывать на счет 3600 руб. Платежи осуществляются: а) в начале каждого месяца; б) в конце каждого месяца.

Решение:

a) Платежи осуществляются в начале каждого месяца (форм. \ref{3}).

\begin{equation}\label{3}
FV = PMT \times \left[\dfrac{\left(1+\frac{i}{k}\right)^{n \times k +1}-1}{\frac{i}{k}}-1 \right],
\end{equation}

$ FV = 3600\times \left[\dfrac{\left(1+\frac{0,07}{12}\right)^{22+1}-1}{\frac{0,07}{12}}-1 \right] = 84736,42 \  \text{руб.}   $

б) Платежи осуществляются в конце каждого месяца (форм. \ref{4}).
\begin{equation}\label{4}
FV =PMT \times \dfrac{(1+\frac{i}{k})^{n \times k}-1}{\frac{i}{k}},
\end{equation}

$ FV =3600 \times \dfrac{(1+\frac{007}{12})^{22}-1}{\frac{0,07}{12}} = 84244,99 \  \text{руб.} $

Ответ: при внесении платежей в начале каждого месяца, накопленная сумма составит 84 736 рублей 42 коп., в конце месяца --- 84 244 рубля 99 коп.

\subsection{Фонд возмещения}

Определить, какими должны быть платежи, чтобы к концу 2-го года иметь на счете, приносящем 35\% годовых, 51 тыс. руб. Платежи осуществляются: а) в конце каждого года; б) в конце каждого квартала.

Решение:

a) Платежи осуществляются в конце каждого года (форм. \ref{5}).

\begin{equation}\label{5}
PMT = FV \times \dfrac{i}{(1+i)^n-1},
\end{equation}

$ PMT = 51 \times \dfrac{0,35}{(1+0,35i)^2-1} = 21,70 \  \text{руб.}$

б) Платежи осуществляются ежеквартально (форм. \ref{6}).

\begin{equation}\label{6}
PMT = FV \times \dfrac{\frac{i}{k}}{(1+\frac{i}{k})^{n \times k}-1},
\end{equation}

$ PMT = 51 \times \dfrac{\frac{0,35}{4}}{(1+\frac{0,35}{4})^{2 \times 4}-1} = 4,67\  \text{руб.} $

Ответ: при ежегодном внесении размер платежа составит 21 700 рублей, при ежеквартальном внесении --- 4 670 рублей.

\subsection{Взнос на амортизацию единицы}

Кредит в размере 210000 руб. выдан на 2 года под 21\% годовых. 
Определить размер аннуитетных платежей. Выплаты по кредиту осуществляются в конце каждого месяца.

Решение:

Выплаты осуществляются в конце каждого месяца (форм. \ref{7}).

\begin{equation}\label{7}
PMT =PV \times\dfrac{\frac{i}{k}}{1-\frac{1}{(1+\frac{i}{k})^{n \times k}}},
\end{equation}

$ PMT =210 \times\dfrac{\frac{0,21}{12}}{1-\frac{1}{(1+\frac{0,21}{12})^{2 \times 12}}} = 10,79\  \text{руб.} $

Ответ: размер платежа составит 10 790 рублей

\subsection{Текущая стоимость аннуитета (платежа)}

Договор аренды квартиры составлен на 18 месяцев. Определить текущую стоимость арендных платежей при 35\% ставке дисконтирования. Арендная плата вносится: а) в размере 1200 руб. в начале каждого полугодия; б) в размере 1200 руб. в конце каждого полугодия.

Решение:

a) Арендная плата в размере 1200 руб. выплачивается в начале каждого полугодия (форм. \ref{8}).
\begin{equation}\label{8}
PV = PMT \times \left[\dfrac{1-\dfrac{1}{\left(1+\dfrac{i}{k}\right)^{n \times k -1}}}{\dfrac{i}{k}}+1\right],
\end{equation}

$PV = 1200 \times \left[\dfrac{1-\dfrac{1}{\left(1+\dfrac{0,35}{2}\right)^{1,5 \times 2 -1}}}{\dfrac{0,35}{2}}+1\right] = 3090,45\  \text{руб.} $

б) Арендная плата в размере 1200 руб. выплачивается в конце каждого полугодия (форм. \ref{9}).

\begin{equation}\label{9}
PV = PMT \times \dfrac{1-\dfrac{1}{\left(1+\dfrac{i}{k}\right)^{n \times k}}}{\dfrac{i}{k}},
\end{equation}

$ PV = 1200 \times \dfrac{1-\dfrac{1}{\left(1+\dfrac{0,35}{2}\right)^{1,5 \times 5}}}{\dfrac{0,35}{2}} = 2630,17\  \text{руб.} $

Ответ: текущая стоимость арендных платежей при выплате в начале полугодия составит 3090 рублей 45 коп., а при выплате в конце полугодия --- 2630 рублей 17 коп.

